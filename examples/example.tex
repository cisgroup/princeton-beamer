\documentclass[aspectratio=169]{beamer}
\usefonttheme{professionalfonts}
\usepackage[T1]{fontenc}
\usepackage{annotate-equations}
% \usepackage{polyglossia}
% \usepackage{amsmath}
% \usepackage{amssymb}
% \usepackage{mathtools}
% \usepackage[math-style=ISO, bold-style=ISO, sans-style=italic, nabla=upright, partial=upright]{unicode-math}
% \setmathfont{Latin Modern Math}
\newlength{\charAheight}
\settoheight{\charAheight}{A}
\usetheme[]{princeton}

% \usepackage{blindtext}
% \usepackage[export]{adjustbox}


\title{The \texttt{princton} beamer theme}

\subtitle{Example Presentation}

\date{\today}

\author[Hackl]{Prof. Dr. Jürgen Hackl}

\institute[PU]{Princeton University}

\authortitle{Dipl.-Ing. Dipl.-Ing. Dr.sc. ETH Z\"urich}

\location{Princeton University}

\event{Example Presentation}


\authorfooter{Jürgen Hackl}
\titlefooter{Example Slide Deck}
\affiliationfooter{Princeton University}
\eventfooter{Introduction to \texttt{pinceton} beamer}
\datefooter{\today}


\coverpicture{./test169.png}

\begin{document}

\maketitle

\section{Overview}

\begin{frame}[t]{Overview}
  This presentation provides an overview of how to use the \texttt{princeton} beamer theme. 
  It also serves as a unit test for developing further extensions to this package.
  \begin{itemize}
  \item Introduction to the \texttt{princeton} theme structure
  \item Demonstration of available layout and style options
  \item Guidance for slide layouts
  \item Validation of functionality through example slides
  \end{itemize}
\end{frame}


\section{Example Slides}

\begin{frame}[t]{Example Slide \multipage{1/2}}{How a slide could look like with the \texttt{princeton} beamer theme}

  This is a generic slide with some text to show how a potential slide could look like.

  \begin{columns}[T]
    \begin{column}{.6\textwidth}

      \begin{block}{Block Info}
        Block elements are useful to show constraint information on a topic. The template supports \alert{block}, \alert{exampleblock} and \alert{alertblock}.
      \end{block}

      \begin{exampleblock}{Exampleblock}
        Blocks can also have items or any other latex environment
        \begin{itemize}
        \item itemize
        \item enumerate
        \item description
        \end{itemize}
      \end{exampleblock}

      \begin{alertblock}{Alertblock}
        To highlight important content use the \alert{alert} command instead of \textbf{textbf}!
      \end{alertblock}

    \end{column}

    \begin{column}{.4\textwidth}
      \centering
      \scriptsize
      \renewcommand{\eqnannotationfont}{\sffamily\tiny}
      \begin{equation*}
        G_{M}(
        \eqnmarkbox[princeton@color@orange]{nVM}{V_{M}}
        ,
        \eqnmarkbox[princeton@color@blue]{nEM}{E_{M}}
        )
      \end{equation*}
      \annotate[yshift=-0em]{above,left, label below}{nVM}{vertices}
      \annotate[yshift=-0em]{above,label below}{nEM}{edges}
      
      \includegraphics[width=\textwidth]{example-image-a}

    \end{column}
  \end{columns}

  \footlineextra{Add references, images sources or other secondary information into the footer!}
\end{frame}



\begin{frame}[t]{Example Slide with a very long title why is the thing not \multipage{2/2}}{Two column 50/50 split with images}

  This is a generic slide with some text to show how a potential slide could look like.

  \begin{columns}[T]
    \begin{column}{.6\textwidth}

      \begin{block}{Block Info}
        Block elements are useful to show constraint information on a topic. The template supports \alert{block}, \alert{exampleblock} and \alert{alertblock}.
      \end{block}

      \begin{exampleblock}{Exampleblock}
        Blocks can also have items or any other latex environment
        \begin{itemize}
        \item itemize
        \item enumerate
        \item description
        \end{itemize}
      \end{exampleblock}

      \begin{alertblock}{Alertblock}
        To highlight important content use the \alert{alert} command instead of \textbf{textbf}!
      \end{alertblock}

    \end{column}

    \begin{column}{.4\textwidth}
      \centering
      \scriptsize
      \renewcommand{\eqnannotationfont}{\sffamily\tiny}
      \begin{equation*}
        G_{M}(
        \eqnmarkbox[princeton@color@orange]{nVM}{V_{M}}
        ,
        \eqnmarkbox[princeton@color@blue]{nEM}{E_{M}}
        )
      \end{equation*}
      \annotate[yshift=-0em]{above,left, label below}{nVM}{vertices}
      \annotate[yshift=-0em]{above,label below}{nEM}{edges}
      
      \includegraphics[width=\textwidth]{example-image-a}

    \end{column}
  \end{columns}

  \footlineextra{Add references, images sources or other secondary information into the footer!}
\end{frame}




\end{document}


%%% Local Variables:
%%% mode: latex
%%% TeX-master: t
%%% End:
